% Configuracion del documento
\documentclass[12pt]{article}
\usepackage{geometry}
 \geometry{
 a4paper,
 total={155mm,242mm}, % para margen derecho e inferior de 25mm
 left=30mm,
 top=30mm,
 }
\usepackage{times}
\usepackage{setspace}
\spacing{1.5}
\usepackage{titlesec}
\titleformat*{\section}{\normalsize\bfseries}


\begin{document}

    \textbf{AVANCES DE INVESTIGACIÓN DE LA TESIS}
    
    Rossi, Sebastián.
    
    Instituto Nacional de Tecnología Industrial.
    
    \section*{Resumen}
    
    Se presenta el avance.
    
    \textbf{Palabras claves:} 
        
    \section{Introducción}
    
    El Instituto Nacional de Tecnología Industrial (INTI) \cite{inti-web} ofrece servicios de asistencia técnica a la industria argentina. Desde 2014 se formó el equipo que hoy trabaja en el Departamento de Ingeniería de Productos Industriales de la sede Rosario. La demanda de empresas del sector de fabricantes de maquinaria agrícola dio lugar al desarrollo de capacidades internas para la experimentación y evaluación de partes y mecanismos de los productos que fabrican. 
    
    Motivos de las empresas: dificultades de importación. Falta de información, desconocimiento de los productos importados.
    
    El INTI cuenta con un plan de capacitación de sus agentes, contemplando la realización de carreras de posgrado. La carrera de Doctorado en Ingeniería del agente Sebastián Rossi está aprobada por la gerencia de recursos humanos de la institución.
    
    Previo a la formación del equipo de trabajo, Gastón Bourges trabajó en modelado y simulación de sistemas de distribución de semillas por aire, llamados air-drill, y al incorporarse al INTI en 2013 completó su tresis doctoral. Los trabajos de investigación que quedaron planteados, principalmente en el campo experimental, y las necesidades de desarrollar sistemas de medición y evaluación para sistemas de siembra dentro del INTI, llevaron al agente Sebastián Rossi a iniciar un proyecto de tesis doctoral.
    
    Cursos y plan de tesis.
    
    \section{Perspectiva teórica}
    
    
    
    \section{Materiales y métodos}

    Tipos de dosificadores (en into?) 
    experimentos realizados
    equipamiento utilizado
    herramientas para análisis de datos    
    
    
    \section{Resultados y discusiones}
    
    Comentarios sobre resultados
    publicaciones hechas
    trabajos planificados
    
    
	\section{Conclusiones}
	
	\section{Para bibliografía}
	
	https://www.argentina.gob.ar/inti
	
    https://www.argentina.gob.ar/servicios-inti/maquinaria-agricola
    
\end{document}


