% Configuracion del documento
\documentclass[12pt]{article}
\usepackage{geometry}
 \geometry{
 a4paper,
 total={155mm,242mm}, % para margen derecho e inferior de 25mm
 left=30mm,
 top=30mm,
 }
\usepackage{times}
\usepackage{setspace}
\spacing{1.5}
\usepackage{titlesec}
\titleformat*{\section}{\normalsize\bfseries}
\titleformat*{\subsection}{\normalsize\bfseries}

\begin{document}

    \textbf{AVANCES DE INVESTIGACIÓN DE LA TESIS}
    
    Rossi, Sebastián.
    
    Instituto Nacional de Tecnología Industrial.
    
    \section*{Resumen}
    
    Se presenta el avance.
    
    \textbf{Palabras claves:} 
        
    \section{Introducción}
    
    El Instituto Nacional de Tecnología Industrial (INTI) \cite{inti-web} ofrece servicios de asistencia técnica a la industria argentina. Desde 2014 se formó el equipo de trabajo que hoy se desempeña en el Departamento de Ingeniería de Productos Industriales (DIPI) de la sede INTI Rosario. La demanda de empresas del sector de fabricantes de maquinaria agrícola dio lugar al desarrollo de capacidades internas para la experimentación y evaluación de partes y mecanismos de los productos que fabrican. 
    
    Las restricciones en la importación de dispositivos tecnológicos, que limitan el acceso a insumos críticos, la escasez de información técnica sobre partes fabricadas por terceros y el crecimiento de las empresas y expansión de sus departamentos de ingeniería, son algunas de las razones por las que los fabricantes empezaron a desarrollar sus propias partes y componentes tecnológicos.
    
    En 2023 se formalizó a través de una disposición del INTI el Programa de Maquinaria Agrícola \cite{programa-maq-agricola-web}. Este programa  tiene como objetivo fortalecer la articulación entre instituciones del sistema científico y empresas del sector industrial, mediante la conformación de un consejo asesor y la creación de un observatorio sobre maquinaria agrícola.
    
    El INTI cuenta con un plan de capacitación de sus agentes, contemplando la realización de carreras de posgrado. La carrera de Doctorado en Ingeniería del agente Sebastián Rossi está aprobada por la gerencia de recursos humanos de la institución.
    
    Previo a la formación del DIPI, Gastón Bourges trabajó en modelado y simulación de sistemas de distribución de semillas por aire, llamados air-drill, y luego de incorporarse al INTI en 2013 completó su tesis doctoral. Los trabajos de investigación que quedaron planteados, principalmente en el campo experimental, y las necesidades de desarrollar sistemas de medición y evaluación para sistemas de siembra para que el INTI brinde el servicio, sirvieron de base al agente Sebastián Rossi para iniciar su proyecto de tesis doctoral.
    
    En 2017 comenzó a realizar cursos acreditables como materias de doctorados. En 2019 presento el plan de tesis y luego, en 2020 fue admitido por la escuela de posgrado de la Facultad de Ciencias Exactas, Ingeniería y Agrimensura de la Universidad Nacional de Rosario como alumno de doctorado.
    
    En este trabajo se presentan las tareas realizadas y el grado de avance de la tesis del doctorando.
    
    \section{Perspectiva teórica}
    
    
    
    \section{Materiales y métodos}
	
	% Esto va en la intro? separar la primer oracion
    En el desarrollo de la tesis se trabajó en dos sistemas involucrados en las sembradoras: el transporte de semillas desde una tolva central a cada uno de los cuerpos de siembra\footnote{El cuerpo de siembra posee discos que abren el surco, tubos por donde se descarga fertilizante y semillas y, en su parte posterior, ruedas que tapan el surco.} mediante mangueras con flujo de aire forzado y dosificadores de precisión, que son mecanismos que se montan uno por cuerpo de siembra, poseen una pequeña tolva que les sirve de almacenamiento intermedio de semillas y las va dejando caer de una a la vez, con el fin de que queden equiespaciadas en el surco. Algunas sembradoras pueden contar con ambos sistemas, aunque existen sistemas air-drill que dejan caer las semillas tal como llegan por la manguera (siembra a chorrillo) o sembradoras con dosificadores de precisión que reciben las semillas directamente de tolvas repartidas en toda la sembradora.
    
	\subsection{Evaluación de sistemas de siembra}
	
	La norma ISO 7256 posee 2 partes y establece cómo se debe evaluar tanto la dosificación de precisión (parte 1) como la dosificación a chorrillo (parte 2), en función de las distancias entre semillas consecutivas medidas en el surco.
	
	La experimentación a campo, haciendo funcionar la sembradora en condiciones normales, resulta costosa, demanda mucho tiempo tanto la ejecución como la obtención de datos y está sujeta a las buenas condiciones climáticas. Si bien son necesarias para el testeo de un prototipo en su etapa final de desarrollo, en etapas previas resulta conveniente que los experimentos se lleven a cabo en bancos de ensayo. Estos dispositivos están siempre en el mismo lugar, por lo que para obtener la separación entre semillas se puede recurrir a una plataforma móvil que simula el movimiento del suelo o directamente estimarse registrando el instante de paso de cada semilla por un lugar y calcularse la distancia teórica suponiendo una velocidad de avance constante. En el desarrollo de la tesis se utiliza la última opción como método de evaluación.
	Por esta razón, una parte del trabajo se enfocó en el desarrollo y evaluación de técnicas de sensado que detectan y registran el instante de paso de cada semilla.	
	
	\subsection{Desarrollo de sensores}
    
    El primer sensor estudiado fue el de placa de impacto con transductor piezoeléctrico. Las primeras pruebas exploratorias se hicieron con placas de fibra de densidad media (MDF) con melamina en ambas caras y un micrófono piezoeléctrico tipo disco de 20 mm de diámetro adherido con cinta bifaz. La adquisición de datos se hizo simplemente grabando audio con una computadora para registrar la señal eléctrica del sensor. Con estas señales se comenzó a programar el método para detectar el tiempo de impacto de cada semilla.
    
    Para la mejora de este sensor se llevaron a cabo pruebas exploratorias con diferentes materiales empleados como placa de impacto y otros como amortiguantes de las vibraciones. Con los resultados obtenidos se escogieron 3 combinaciones de material de placa y amortiguante y se realizó un experimento aleatorizado. 
    
    En una etapa posterior, se comenzó a trabajar en la detección por barrera óptica. Su construcción se basó en técnicas encontradas en la literatura. Utilizando un arreglo lineal de fototransistores se puede obtener tanto información temporal como espacial, útil para conocer la deriva del ángulo con la que salen las semillas de un dosificador o de un tubo de descarga.
	
	Actualmente, se está trabajando en la evaluación de desempeño de una barrera óptica con doble arreglo lineal de sensores ópticos, a 90° en un mismo plano. Con esta configuración se puede conocer las coordenadas ``X Y'' por donde cada semilla atraviesa el sensor.
	
	\subsection{Experimentos realizados}
	En 2017 y 2019 se realizaron dos experimentos en bloques aleatorizados para evaluar los factores que influyen en la distribución de semillas en un cabezal distribuidor de 6 salidas. El fin de estos cabezales es distribuir en forma equitativa el flujo de semillas que le llega entre todas sus salidas. Esta distribución equitativa se cumple en buena medida con semillas pequeñas, pero se encuentran diferentes proporciones de flujos de salida en semillas de tamaño mayor, como es el caso de la soja. 
	
	En 2022 se realizó un experimento con semillas de maíz, soja y girasol para evaluar el desempeño de un sistema de detección de semillas por impacto. Cada ensayo se filmó con cámara de alta velocidad en la zona de sensado. A partir de la inspección de los videos, se hizo un registro manual de cada semilla. Los resultados de detección se contrastaron con registros manuales para contabilizar los falsos positivos y falsos negativos.
	
	En 2023 se realizó un experimento para evaluar un sensor de barrera óptica en 1 dimensión. 
	
	Luego, En 2025 se realizó un experimento para evaluar un sensor de barrera óptica en 2 dimensiones. El proceso de validación se hizo de manera similar al del sensor por placa de impacto, aunque para el arreglo en 2 dimensiones se agregó un espejo para filmar desde 2 ángulos al mismo tiempo.
    
    
    \subsection{Equipamiento utilizado}
    Los dispositivos utilizados para detectar semillas son, por un lado, los que están bajo evaluación: sensor piezoeléctrico y par de LED y fototransistor infrarojos, y por otro lado el utilizado como patrón: la cámara de alta velocidad Fastec TS5-S. Para la adquisición de datos se utilizaron las placas de sonido de computadora (integradas y externas USB), un adquisidor de datos HBM QuantumX MX1516B y adaptadores de tarjeta micro SD. Para procesamiento en tiempo real de las señales de sensores se utilizaron placas de desarrollo con microcontrolador ESP32. 
    
    Para procesar señales y analizar los datos se utilizó python con las bibliotecas numpy, pandas y scipy, y R en el entorno RStudio.
    
    
    
    
    \section{Resultados y discusiones}
    
    La evaluación del sistema de sensado de semillas con placa mostró resultados positivos. Los errores de detección para semillas dosificadas a velocidades bajas se mantuvieron debajo del 1\%. Para semillas de soja a alta velocidad se acercó al 5\%. Su desempeño permite obtener resultados representativos del comportamiento de un dosificador de precisión. El bajo costo de los materiales y su directa conexión a la entrada de audio de una computadora hace que cualquiera pueda usarlo tanto para desarrollo de un nuevo dosificador como para puesta a punto previo a una jornada de siembra.
    
    El sistema de arreglo de sensores ópticos requiere de un circuito electrónico de alimentación y adaptación de las señales y de un microcontrolador para procesar los cambios de estado de los sensores. Si bien esto agrega una complejidad respecto a la placa de impacto, todos los componentes empleados son de bajo costo. Los errores de detección son inferiores en el caso de soja a alta velocidad respecto a la placa de impacto y como ventaja se agrega la información espacial, es decir, por dónde pasa cada semilla en el plano de sensado.
    
    Ambos sistemas de sensado se encuentran en una etapa de prototipo. Se deberá trabajar en diseño electrónico para mejorar inmunidad al ruido, en diseño de carcasas para que puedan utilizarse en un ambiente fuera del laboratorio. Otro trabajo pendiente es la transferencia formal de estos conocimiento al sector industrial. En 2023 se proyectó hacer la transferencia del sensado de placa a la empresa Siembra Neumática, adaptándolo a sus bancos de pruebas. Este trabajo se iba a realizar en un proyecto con financiación y se dio de baja antes de que comience.
    
    El sistema de sensado por placa de impacto se utilizó para contabilizar las semillas por cada salida del cabezal distribuidor en los experimentos realizados en el sistema air-drill. En este trabajo se encontró que uno de los factores que más influyen en la distribución es la longitud relativa entre las mangueras de salida. Se planea hacer un nuevo experimento para indagar con mayor profundidad el efecto de las características geométricas de las mangueras de salida sobre las proporciones de semillas por cada una de ellas.

    publicaciones hechas
    
    
	\section{Conclusiones}
	
	\section{Para bibliografía}
	

	
    
    
    \begin{thebibliography}{8}

	\bibitem{inti-web}
    https://www.argentina.gob.ar/inti

    \bibitem{programa-maq-agricola-web}
    https://www.argentina.gob.ar/servicios-inti/maquinaria-agricola
    
    \bibitem{iso7256} Standard, I. S. O. 7256/1-1984 : ``Sowing equipment-Test methods Part 1: Single seed drills (precision drills)''. Geneva, Switzerland: International Organization for Standardization, 1984
    \end{thebibliography}
    
\end{document}


