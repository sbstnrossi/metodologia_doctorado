% Configuracion del documento
\documentclass[12pt]{article}
\usepackage{geometry}
 \geometry{
 a4paper,
 total={155mm,242mm}, % para margen derecho e inferior de 25mm
 left=30mm,
 top=30mm,
 }
\usepackage{times}
\usepackage{setspace}
\spacing{1.5}
\usepackage{titlesec}
\titleformat*{\section}{\normalsize\bfseries}


\begin{document}

    \textbf{AVANCES DE INVESTIGACIÓN DE LA TESIS}
    
    Rossi, Sebastián.
    
    Instituto Nacional de Tecnología Industrial.
    
    \section*{Resumen}
    
    Se presenta el avance.
    
    \textbf{Palabras claves:} 
        
    \section{Introducción}
    
    Qué hace el INTI. 
    El Instituto Nacional de Tecnología Industrial (INTI) \cite{inti-web} ofrece servicios de asistencia técnica a la industria argentina. Desde 2014 se formó el equipo que hoy trabaja en el Departamento de Ingeniería de Productos Industriales de la sede Rosario. La demanda de empresas del sector de fabricantes de maquinaria agrícola dio lugar al desarrollo de capacidades internas para la experimentación y evaluación de partes y mecanismos de los productos que fabrican. 
    
    https://www.argentina.gob.ar/inti
    https://www.argentina.gob.ar/servicios-inti/maquinaria-agricola
    
    Demanda de empresas.
    Planes de capacitación. 
    Gastón empezó con air-drill. 
    Cursos y plan de tesis.
    
    \section{Perspectiva teórica}
    
    
    
    \section{Materiales y métodos}

    Tipos de dosificadores (en into?) 
    experimentos realizados
    equipamiento utilizado
    herramientas para análisis de datos    
    
    
    \section{Resultados y discusiones}
    
    Comentarios sobre resultados
    publicaciones hechas
    trabajos planificados
    
    
	\section{Conclusiones}
	
	    
    
\end{document}


